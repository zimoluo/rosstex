\section{Useful Conclusions}

The conclusions listed below are quite useful for grinding problem sets, but sadly they are not permitted to use unless first proven – that's why we have them right here.

\begin{theorem**}[Induction]
For every statement $P(n)$ where $n \in \Z^+$, if $P(1)$ is true and $P(s)\implies P(s+1)$ for some $s \in \Z^+$, then the statement $P(n)$ is true for every $n \in \Z^+$.
\end{theorem**}
\begin{proof}
    Suppose $P(n)$ is a statement such that $P(1)$ is true and $P(s)\implies P(s+1)$ for some $s \in \Z^+$.
    
    Let $S = \left\{ a\in \Z^+ \vert P(a) \text{ is false} \right\} \subseteq \Z^+$. By \textit{WOP}, there exists $l \in S$ such that $l$ is the least element in $S$. Since $P(1)$ is true, $1 \notin S$, so $l \neq 1$. By \textit{OLE}, $1$ is the least element in $\Z^+$ overall. It thus follows $l > 1$. This implies $l-1 \in \Z^+$. But since $l$ is the least element of $S$, $l-1 \notin S$. Therefore, $P(l-1)$ is true. Hence, by definition of $P(n)$, $P(l-1+1) = P(l)$ is also true. This contradicts the fact that $l \in S$, implying that $S = \emptyset$. Therefore, $P(n)$ is true for all $n \in \Z^+$
\end{proof}
%%%%%%%%%%%%%%%%%%%%%%%%%%%%%%%%%%%%%%%%%%%%%%%%%%%%%%%%%%%%%%%%%
\begin{theorem**}[Division Algorithm]
For $a, b \in \Zplus$, we can write
\begin{align*}
    a &= bq+r \text{\quad for $r, q \in \Zplus, 0\leq r < b$}.
\end{align*}
\end{theorem**}
\begin{proof}
    Consider $S = \left\{bq+r\ \vert\ \forall r \in \Zplus, 0\leq r < b\right\}$. We will show $a \in S$.

    Now, say $B = \left\{ a: a \notin S \right\}$ is a non-empty subset of $\Zplus$. Since $B\subseteq \Zplus$, by \textit{WOP}, $B$ has a minimal element $1 \notin B$, because $1 = 0\times 1 + 1$. Thus $l$, the least element of $B$, is greater than $1$. Note that if $x \in S$, so is $x+1$.

    Thus consider $l - 1$. Since $l$ is the least value of $B$, $l-1 \notin B$ because $l-1 < l$. But if $l-1 \in S$, then $l \in S$ as well. This contradicts the fact that $l$ is the least element of $S$, implying that $S = \emptyset$.
\end{proof}

\begin{theorem**}[Bezout's Identity]
    For $a, b \in \Z$, we can express $\gcd(a, b)$ as an integer linear combination of $a$ and $b$. That is, there exists integer solutions for
    \begin{equation*}
        ax+by=\gcd(a, b) .
    \end{equation*}
\end{theorem**}
\begin{proof}
    Consider the equation $s = ax+by$, where $s\in \Zplus$. Let $S \subseteq \Zplus$ be the non-empty set of positive integers of solutions for $ax + by$.

    Consider $l$, the least element of $S$. We thus have
    \begin{equation*}
        ax+by = l
    \end{equation*}

    Now, apply the division algorithm to $a$ and $l$.
    \begin{align*}
        a &= ql + r, \quad 0\leq r < l \\
        a &= qax + qby + r \\
        r &= a(1-qx) - b(qy) \\
    \end{align*}
    Rearranging the equation, we find that $r$ also satisfy the linear combination of $a$ and $b$. But since $r < l$, in order not to contradict the fact that $l$ is the least element of $S$, we must have $r = 0$.

    Thus $l \mid a$. By a similar argument we can also show that $l \mid b$. Thus $l$ is a common factor of $a$ and $b$. Consider $d = \gcd(a, b)$. In Set \#3 Problem 11 we've shown that $l \mid d$. Since $d \mid a$ and $d \mid b$, we have that $d$ divides any linear combination of $a$ and $b$, which includes $l$. Because $d \mid l$ and $l \mid d$, it follows that $d = l$. Thus there exists integer solutions for the equation
    \begin{equation*}
        ax+by = \gcd(a, b) .
    \end{equation*}
\end{proof}

\begin{theorem**}
    $\gcd(m, n) = 1$ implies $\gcd(mn, m+n) = 1$
\end{theorem**}
\begin{proof}
    Consider $m, n$ where $\gcd(m, n) = 1$. Assume $\gcd(mn, m+n) > 1$. Then there must be some $p$ that $\gcd(mn, m+n)$. Then $p \mid mn$. Thus $p \mid m$ or $p \mid n$. But since $p \mid m+n$, if $p$ divides one of $m$ and $n$, then it also divides the other one. But if $p \mid m$ and $p \mid n$, then $p \leq \gcd(m, n) = 1$, which is not possible. Thus $\gcd(m, n)$ implies $\gcd(mn, m+n) = 1$.
\end{proof}

\begin{theorem**}[{Division Algorithm in $\Z_m[x]$}]

    Division algorithm applies in $\Z_m[x]$.

\end{theorem**}

    \begin{proof}
    Let us define $S$ to be the set of polynomials $r(x)$ with degree $n$ with $r(x) = f(x) - q(x) \cdot g(x)$.\\\\
    We first show that S is non empty. Simply taking $q(x)$ gives $r(x) = f(x)$, which is valid. Therefore $r(x)=f(x)$ must be in the set S.\\\\
    If $0$ is in $S$, then we are done, since we can take $r(x)=0$ and $f(x) = q(x) \cdot g(x)+r(x)$. Therefore let $0 \not \in S$. Since degrees are nonnegative integers, without 0, it must be positive. We can therefore apply WOP to $S$ and get a polynomial $r_l(x)$ in S with minimal degree and its associated $q_l(x)$. By definition, $\deg(r_l(x)) >  \deg(g(x))$. Let the leading coeffecient of $r_l(x)$ be $L$, and $g(x)$ be $G$. Since $G$ is a unit in $m$, let the inverse of $G$ mod m be $G^{-1}$. We then know $L \equiv L \cdot (G \cdot G^{-1}) \bmod m$. \\\\
    Assume for contradiction that $r_l(x)$ has degree $n \geq \deg(g(x))$. Now consider the polynomial $p(x) = r_l(x)-(L \cdot G^{-1})(x^{\deg(r_l(x))-\deg(g(x))})g(x)$. Note that $\deg(p(x)) < \deg(r_l(x))$, since the leading term of $r_l(x)$ is cancelled. But $p(x)$ is also in the set $S$, since we have $p(x) = f(x) - (r_q(x) +(L \cdot G^{-1})(x^{\deg(r_l(x))-\deg(g(x))}) )g(x)$. This raises a contradiction, since by WOP we assumed $r_l(x)$ has the minimal degree. Therefore, there exist a $r(x)$ with degree $n$ such that $0 \geq n < \deg(g(x))$.
\end{proof}

\end{document}
